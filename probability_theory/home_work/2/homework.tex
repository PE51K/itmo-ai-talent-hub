\documentclass{article}

\usepackage[utf8]{inputenc}
\usepackage{amsmath}
\usepackage{amsfonts}
\usepackage{amssymb}

\title{Probability Theory Homework 2}
\author{Gregory Matsnev}
\date{\today}

\begin{document}

\maketitle

\section{Problem 1}

\textbf{Problem.} Two shooters take turns shooting at a target until the first hit. The probability of hitting for the first shooter is $0.2$, and for the second is $0.3$. What is the probability that the first shooter will make more shots than the second?

\textbf{Solution.}
Let $p$ be the probability that the \emph{first} shooter is the one to score the first hit (equivalently: that he makes more shots than the second). After any full round in which both miss, the game resets to the initial state (first shooter to act, same accuracies), so
\[
p \;=\; p_1 \;+\; (1-p_1)(1-p_2)\,p,
\]
where $p_1=0.2$ and $p_2=0.3$. Solving,
\[
p \;=\; \frac{p_1}{1-(1-p_1)(1-p_2)}
= \frac{0.2}{1-0.8\cdot0.7}
= \frac{0.2}{0.44}
= \frac{5}{11}.
\]
(Equivalently, \(p=\sum_{k\ge0}\bigl((1-p_1)(1-p_2)\bigr)^k p_1\) is a geometric series.)

Therefore, the probability that the first shooter makes more shots than the second is
\[
\boxed{\dfrac{5}{11}}.
\]

\section{Problem 2}

\textbf{Problem.} There are 5 white and 6 black balls in the first urn, and 4 white and 8 black balls in the second urn. Three balls are drawn at random from the first urn and placed into the second urn. After this, four balls are drawn at random from the second urn. Find the probability that all four balls drawn from the second urn are white. Hint: remember about conditional probability!

\textbf{Using your notation.}
Let $F_k=\{f=k\}$ be the event that $k$ white balls are transferred ($k=0,1,2,3$),
and let $G=\{g(4)\}$ be the event that all four drawn from urn 2 are white.
By the law of total probability,
\[
\Pr(G)=\sum_{k=0}^3 \Pr(G\mid F_k)\Pr(F_k).
\]

\textit{Transfer step.} From urn 1 (5W,6B) to urn 2, the number $f$ of whites moved is
hypergeometric:
\[
\Pr(F_k)=\frac{\binom{5}{k}\binom{6}{3-k}}{\binom{11}{3}},\qquad k=0,1,2,3.
\]

\textit{Draw from urn 2.} After moving $k$ whites, urn 2 has $4+k$ whites and $11-k$ blacks
(total $15$). Thus
\[
\Pr(G\mid F_k)=\frac{\binom{4+k}{4}}{\binom{15}{4}}.
\]

Therefore
\[
\Pr(G)=\sum_{k=0}^{3}\frac{\binom{5}{k}\binom{6}{3-k}}{\binom{11}{3}}
\cdot \frac{\binom{4+k}{4}}{\binom{15}{4}}
= \frac{47}{6435}\approx 0.00730.
\]

\[
\boxed{\Pr(\text{4 whites from urn 2})=\dfrac{47}{6435}}
\]

\section{Problem 3}

\textbf{Problem.} An insurance company categorizes its policyholders into risk classes: Class I (low risk), Class II (medium risk), and Class III (high risk). Among these clients, 50\% are in the first class, 30\% in the second, and 20\% in the third. The probability of an insurance payout being necessary for the first risk class is 0.01, 0.03 for the second, and 0.08 for the third. What is the probability that a policyholder who received a monetary payout belongs to the low-risk group?

\textbf{Solution (matching your notation and steps).}

\textit{Set-up.} 
Let the random variable $k\in\{1,2,3\}$ indicate the risk class (1 = low, 2 = medium, 3 = high).
Let $t\in\{0,1\}$ indicate whether a payout occurs ($t=1$) or not ($t=0$).

Given:
\[
P(k=1)=0.5,\qquad P(k=2)=0.3,\qquad P(k=3)=0.2,
\]
\[
P(t=1\mid k=1)=0.01,\qquad P(t=1\mid k=2)=0.03,\qquad P(t=1\mid k=3)=0.08.
\]

\textit{Goal.} Compute $P(k=1\mid t=1)$.

\medskip
\textit{Your chain (Bayes).}
Start from
\[
P(k=1\mid t=1)=\frac{P(k=1\wedge t=1)}{P(t=1)}
=\frac{P(t=1\mid k=1)\,P(k=1)}{P(t=1)}.
\]

\textit{Total probability for the denominator.}
\[
P(t=1)=\sum_{r=1}^{3} P(t=1\mid k=r)\,P(k=r)
=0.01\cdot0.5+0.03\cdot0.3+0.08\cdot0.2
=0.005+0.009+0.016=0.03.
\]

\textit{Plug in.}
\[
P(k=1\mid t=1)
=\frac{0.01\cdot 0.5}{0.03}
=\frac{0.005}{0.03}
=\frac{1}{6}\approx 0.1667.
\]

\[
\boxed{P(\text{low risk}\mid \text{payout})=\dfrac{1}{6}\ (\approx 16.7\%).}
\]

\section{Problem 4}

\textbf{Problem.} The cloakroom of a theater has randomly permuted all \( n \) visitors’ hats. Find the probability that at least one visitor gets his hat. Give a formula answer with derivation. Given \( n = 4 \), give a numerical answer. (Hint: use the inclusion–exclusion formula.)

\textbf{Solution (following your outline).}
Let $t$ be the number of visitors who receive their own hat. Then
\[
\mathbb{P}(t\ge 1)=1-\mathbb{P}(t=0).
\]
Thus we only need $\mathbb{P}(t=0)$, the probability that \emph{no one} gets his own hat
(i.e., a derangement).

There are $n!$ total permutations. For $r\in\{0,1,\dots,n\}$, fix a set $S$ of $r$
people and force them to get their own hats; the remaining $n-r$ hats can be
permuted in $(n-r)!$ ways. By inclusion–exclusion, the number of permutations
with \emph{no} fixed points is
\[
D_n
= \sum_{r=0}^n (-1)^r \binom{n}{r} (n-r)!.
\]
Dividing by $n!$ gives
\[
\mathbb{P}(t=0)=\frac{D_n}{n!}
=\sum_{r=0}^n \frac{(-1)^r}{r!}.
\]
Therefore
\[
\mathbb{P}(t\ge 1)=1-\mathbb{P}(t=0)
=1-\sum_{r=0}^n \frac{(-1)^r}{r!}
=\sum_{r=1}^n \frac{(-1)^{r+1}}{r!}.
\]

\textbf{For $n=4$:}
\[
\mathbb{P}(t\ge 1)=\sum_{r=1}^4 \frac{(-1)^{r+1}}{r!}
=1-\frac{1}{2!}+\frac{1}{3!}-\frac{1}{4!}
=1-\frac12+\frac16-\frac{1}{24}
=\frac{5}{8}=0.625.
\]

\textit{Remark.} The naive product
$\frac{(n-1)(n-2)\cdots 1}{n!}=\frac{1}{n}$ counts the event
“person 1 is not fixed,” not “no one is fixed.” These events are dependent;
inclusion–exclusion corrects for this dependence.

\section{Problem 5}

\textbf{Problem.} If you get a positive result on a COVID test that only gives a false positive with probability \( 0.001 \) (true positive with probability \( 0.999 \)), what’s the chance that you’ve actually got COVID, if:

a) the prior probability that a person has COVID is \( 0.01 \);

b) the prior probability that a person has COVID is \( 0.0001 \).

\[
\textbf{Solution.}
\]

Let:
\[
P(C) = \text{prior probability of having COVID}, \quad
P(\neg C) = 1 - P(C),
\]
\[
P(T^+|C) = \text{probability of a positive test if infected (true positive)} = 0.999,
\]
\[
P(T^+|\neg C) = \text{probability of a positive test if not infected (false positive)} = 0.001.
\]

We want:
\[
P(C|T^+) = \frac{P(T^+|C)P(C)}{P(T^+|C)P(C) + P(T^+|\neg C)P(\neg C)}.
\]

---

\textbf{(a) Prior probability \( P(C) = 0.01 \).}

\[
P(\neg C) = 1 - 0.01 = 0.99.
\]
\[
P(T^+|C)P(C) = 0.999 \times 0.01 = 0.00999.
\]
\[
P(T^+|\neg C)P(\neg C) = 0.001 \times 0.99 = 0.00099.
\]
\[
P(T^+) = 0.00999 + 0.00099 = 0.01098.
\]
\[
P(C|T^+) = \frac{0.00999}{0.01098} \approx 0.9098.
\]

\[
\boxed{P(C|T^+) \approx 90.98\%}.
\]

---

\textbf{(b) Prior probability \( P(C) = 0.0001 \).}

\[
P(\neg C) = 1 - 0.0001 = 0.9999.
\]
\[
P(T^+|C)P(C) = 0.999 \times 0.0001 = 0.0000999.
\]
\[
P(T^+|\neg C)P(\neg C) = 0.001 \times 0.9999 = 0.0009999.
\]
\[
P(T^+) = 0.0000999 + 0.0009999 = 0.0010998.
\]
\[
P(C|T^+) = \frac{0.0000999}{0.0010998} \approx 0.0908.
\]

\[
\boxed{P(C|T^+) \approx 9.08\%}.
\]

---

\textbf{Summary Table.}

\[
\begin{array}{c|c|c}
\text{Prior } P(C) & P(C|T^+) & \text{Interpretation} \\ \hline
0.01 & 0.9098 & \text{High posterior probability (90.98\%)} \\
0.0001 & 0.0908 & \text{Much lower posterior probability (9.08\%)} 
\end{array}
\]

Even with a highly accurate test, a very low prior (prevalence) greatly reduces the probability of truly being infected after a positive test.

\section{Problem 6}

\textbf{Problem.} There are two baskets. The first basket contains one white ball, and the second basket contains one black ball. One basket is chosen randomly and a white ball is put into the chosen basket. The balls in this basket are shuffled. Then one ball is extracted from this basket, and it turns out to be white. What is the posterior probability that the second ball drawn from this basket is also white?

\textbf{Solution.}
Let $A$ be the event that the basket chosen to receive the extra white ball is the basket that originally had a white ball (call it Basket 1). Then
\[
\Pr(A)=\Pr(A^c)=\tfrac12.
\]

If $A$ occurs, Basket 1 becomes $(W,W)$, so drawing a white is certain:
\[
\Pr(W\mid A)=1,
\]
and the remaining ball is also white with probability $1$.

If $A^c$ occurs, the other basket (originally $(B)$) becomes $(W,B)$, so
\[
\Pr(W\mid A^c)=\tfrac12,
\]
and \emph{given} that a white was drawn, the remaining ball is black, so the second draw being white has probability $0$.

By Bayes' rule, conditioning on the observed first draw being white,
\[
\Pr(A\mid W)=\frac{\Pr(W\mid A)\Pr(A)}{\Pr(W\mid A)\Pr(A)+\Pr(W\mid A^c)\Pr(A^c)}
=\frac{1\cdot \tfrac12}{1\cdot \tfrac12+\tfrac12\cdot \tfrac12}
=\frac{2}{3}.
\]
Therefore, the posterior probability that the second ball from this basket is also white is
\[
\Pr(\text{second is } W \mid W)=\Pr(A\mid W)\cdot 1+\Pr(A^c\mid W)\cdot 0=\boxed{\tfrac{2}{3}}.
\]

\end{document}
