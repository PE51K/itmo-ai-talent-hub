\documentclass{article}

\usepackage[utf8]{inputenc}
\usepackage{amsmath}
\usepackage{amsfonts}
\usepackage{amssymb}

\title{Probability Theory Homework 6}
\author{Gregory Matsnev}
\date{\today}

\begin{document}

\maketitle

\section{Problem 1}

\textbf{Problem.} Find the PDF of $Y = 1 - X^3$, where $X$ is the random variable distributed according to the Cauchy law, i.\ e.\ with the PDF
\[
\phi(x) = \frac{1}{\pi(1 + x^2)}
\]

\textbf{Solution.}
Let
\[
Y=1-X^3.
\]
Then
\[
y=1-x^3 \quad \Longleftrightarrow \quad x^3=1-y \quad \Longleftrightarrow \quad x=(1-y)^{1/3}.
\]
Computing the CDF:
\[
F_Y(y)=\mathbb P(Y\le y)=\mathbb P(1-X^3\le y)
=\mathbb P(X^3\ge 1-y)=\mathbb P\!\left(X\ge (1-y)^{1/3}\right)
=1-F_X\!\left((1-y)^{1/3}\right).
\]
Differentiating:
\[
f_Y(y)=\frac{d}{dy}F_Y(y)
=-\frac{d}{dy}F_X\!\left((1-y)^{1/3}\right)
=-f_X\!\left((1-y)^{1/3}\right)\frac{d}{dy}(1-y)^{1/3}.
\]
\[
\frac{d}{dy}(1-y)^{1/3}=-\frac{1}{3}(1-y)^{-2/3},
\qquad
\left|\frac{d}{dy}(1-y)^{1/3}\right|=\frac{1}{3|1-y|^{2/3}}.
\]
Hence
\[
f_Y(y)=f_X\!\left((1-y)^{1/3}\right)\frac{1}{3|1-y|^{2/3}}.
\]
Substituting $f_X$:
\[
f_X\!\left((1-y)^{1/3}\right)=\frac{1}{\pi\left(1+\left((1-y)^{1/3}\right)^2\right)}
=\frac{1}{\pi\left(1+|1-y|^{2/3}\right)}.
\]
Therefore
\[
f_Y(y)=\frac{1}{\pi\left(1+|1-y|^{2/3}\right)}\cdot \frac{1}{3|1-y|^{2/3}}
=\boxed{\ \frac{1}{3\pi\,|1-y|^{2/3}\left(1+|1-y|^{2/3}\right)}\ },\qquad y\in\mathbb R.
\]

\section{Problem 2}

\textbf{Problem.} Find the expected value and the variance of the random variable $Y = 2 - 3\sin X$, given that the PDF of $X$ is
\[
\phi(x) = \frac{1}{2}\cos x \text{ for } x \in \left[-\pi/2,\pi/2\right]
\]

\textbf{Solution.}

Letting $f_X(x)=\frac12\cos x$ for $x\in\left[-\frac\pi2,\frac\pi2\right]$.

Computing:
\[
\mathbb E[Y]=\mathbb E[2-3\sin X]
=\int_{-\pi/2}^{\pi/2}(2-3\sin x)f_X(x)\,dx
=\int_{-\pi/2}^{\pi/2}(2-3\sin x)\frac12\cos x\,dx
\]
\[
=\int_{-\pi/2}^{\pi/2}\cos x\,dx-\frac32\int_{-\pi/2}^{\pi/2}\sin x\cos x\,dx
=\Bigl[\sin x\Bigr]_{-\pi/2}^{\pi/2}-\frac32\Bigl[\frac12\sin^2 x\Bigr]_{-\pi/2}^{\pi/2}
=2.
\]

Computing:
\[
\mathbb E[Y^2]=\mathbb E[(2-3\sin X)^2]
=\int_{-\pi/2}^{\pi/2}(2-3\sin x)^2 f_X(x)\,dx
=\int_{-\pi/2}^{\pi/2}(4-12\sin x+9\sin^2 x)\frac12\cos x\,dx
\]
\[
=2\int_{-\pi/2}^{\pi/2}\cos x\,dx-6\int_{-\pi/2}^{\pi/2}\sin x\cos x\,dx+\frac92\int_{-\pi/2}^{\pi/2}\sin^2 x\cos x\,dx.
\]
\[
2\int_{-\pi/2}^{\pi/2}\cos x\,dx=2\Bigl[\sin x\Bigr]_{-\pi/2}^{\pi/2}=4,
\qquad
-6\int_{-\pi/2}^{\pi/2}\sin x\cos x\,dx=-6\Bigl[\frac12\sin^2 x\Bigr]_{-\pi/2}^{\pi/2}=0.
\]
\[
\frac92\int_{-\pi/2}^{\pi/2}\sin^2 x\cos x\,dx
=\frac92\int_{-1}^{1}u^2\,du
=\frac92\Bigl[\frac{u^3}{3}\Bigr]_{-1}^{1}
=3.
\]
\[
\mathbb E[Y^2]=4+0+3=7,
\qquad
\mathrm{Var}(Y)=\mathbb E[Y^2]-\bigl(\mathbb E[Y]\bigr)^2=7-2^2=3.
\]

\section{Problem 3}

\textbf{Problem.} The random variable $X$ is defined on the entire real axis with the probability density $\phi(x)=\frac{1}{2}e^{-|x|}$. Find the probability density of the random variable $Y=X^2$ and its mathematical expectation.

\textbf{Solution.}
Let $Y=X^2$. Then $Y\ge 0$ and for $y>0$ the equation $y=x^2$ has two solutions $x=\sqrt y$ and $x=-\sqrt y$. By the change-of-variables formula,
\[
f_Y(y)=\sum_{x:\,x^2=y} f_X(x)\left|\frac{dx}{dy}\right|
=f_X(\sqrt y)\cdot \frac{1}{2\sqrt y}+f_X(-\sqrt y)\cdot \frac{1}{2\sqrt y},\qquad y>0.
\]
Since $f_X(x)=\frac12 e^{-|x|}$, we have $f_X(\sqrt y)=f_X(-\sqrt y)=\frac12 e^{-\sqrt y}$, hence
\[
f_Y(y)=\frac{e^{-\sqrt y}}{2\sqrt y},\qquad y>0,
\qquad\text{and}\qquad
f_Y(y)=0,\ y<0.
\]
(The density has an integrable singularity at $y=0$.)

For the expectation,
\[
\mathbb E[Y]=\mathbb E[X^2]
=\int_{-\infty}^{\infty} x^2 \cdot \frac12 e^{-|x|}\,dx
=\int_{0}^{\infty} x^2 e^{-x}\,dx
=2.
\]

\section{Problem 4}

\textbf{Problem.} Prove formally that if the correlation coefficient $\rho_{XY}$ of two random variables $X$ and $Y$ is equal in absolute value to one, then there is a linear functional relationship between these random variables.

Remember how to prove that $Cov(X,Y)\leq \sigma_X\sigma_Y$.

\textbf{Solution.}

Assuming $\sigma_X>0$ and $\sigma_Y>0$.

Letting \(\widetilde X=X-\mathbb E[X]\) and \(\widetilde Y=Y-\mathbb E[Y]\).

Rewriting
\[
\rho_{XY}=\frac{\mathrm{Cov}(X,Y)}{\sigma_X\sigma_Y}
=\frac{\mathbb E[\widetilde X\widetilde Y]}{\sqrt{\mathbb E[\widetilde X^2]}\sqrt{\mathbb E[\widetilde Y^2]}}.
\]

Applying Cauchy--Schwarz
\[
\bigl|\mathbb E[\widetilde X\widetilde Y]\bigr|
\le \sqrt{\mathbb E[\widetilde X^2]}\sqrt{\mathbb E[\widetilde Y^2]}.
\]

Using $|\rho_{XY}|=1$ and rewriting
\[
|\rho_{XY}|=1
\Longleftrightarrow
\bigl|\mathbb E[\widetilde X\widetilde Y]\bigr|
=\sqrt{\mathbb E[\widetilde X^2]}\sqrt{\mathbb E[\widetilde Y^2]}.
\]

Considering, for \(t\in\mathbb R\),
\[
0\le \mathbb E[(\widetilde Y-t\widetilde X)^2]
=\mathbb E[\widetilde Y^2]-2t\,\mathbb E[\widetilde X\widetilde Y]+t^2\mathbb E[\widetilde X^2].
\]

Minimizing in \(t\) by differentiating
\[
\frac{d}{dt}\Bigl(\mathbb E[\widetilde Y^2]-2t\,\mathbb E[\widetilde X\widetilde Y]+t^2\mathbb E[\widetilde X^2]\Bigr)
=-2\mathbb E[\widetilde X\widetilde Y]+2t\,\mathbb E[\widetilde X^2],
\]
\[
-2\mathbb E[\widetilde X\widetilde Y]+2t\,\mathbb E[\widetilde X^2]=0
\Longleftrightarrow
t=\frac{\mathbb E[\widetilde X\widetilde Y]}{\mathbb E[\widetilde X^2]}.
\]

Substituting this \(t\) and computing
\[
\mathbb E[(\widetilde Y-t\widetilde X)^2]
=\mathbb E[\widetilde Y^2]-\frac{\mathbb E[\widetilde X\widetilde Y]^2}{\mathbb E[\widetilde X^2]}.
\]

Using
\[
\bigl|\mathbb E[\widetilde X\widetilde Y]\bigr|
=\sqrt{\mathbb E[\widetilde X^2]}\sqrt{\mathbb E[\widetilde Y^2]}
\Longleftrightarrow
\mathbb E[\widetilde X\widetilde Y]^2=\mathbb E[\widetilde X^2]\mathbb E[\widetilde Y^2],
\]
obtaining
\[
\mathbb E[(\widetilde Y-t\widetilde X)^2]
=\mathbb E[\widetilde Y^2]-\frac{\mathbb E[\widetilde X^2]\mathbb E[\widetilde Y^2]}{\mathbb E[\widetilde X^2]}
=0.
\]

Concluding
\[
\mathbb E[(\widetilde Y-t\widetilde X)^2]=0
\Longleftrightarrow
\widetilde Y-t\widetilde X=0 \ \text{a.s.}
\Longleftrightarrow
Y-\mathbb E[Y]=t\,(X-\mathbb E[X])\ \text{a.s.}
\]

Setting \(c=t\) and computing
\[
Y=cX+b\ \text{a.s.},\qquad b=\mathbb E[Y]-c\,\mathbb E[X].
\]

\section{Problem 5}

\textbf{Problem.} The distribution surface (joint PDF) of the two-dimensional random variable $(X,Y)$ is a right circular cone, the base of which is a circle centered at the origin with a unit radius. Outside this circle, the joint PDF of this two-dimensional random variable $(X,Y)$ is zero. Find the joint PDF $f(x,y)$, the marginal PDFs and the conditional PDFs $f_x(y)$ and $f_y(x)$. Are the random variables $X$ and $Y$ dependent and/or correlated?

\textbf{Solution.}

Letting $r=\sqrt{x^2+y^2}$. Writing
\[
f_{X,Y}(x,y)=c(1-r)\mathbf 1_{\{r\le 1\}}.
\]
Normalizing
\[
1=\iint_{\mathbb R^2} f_{X,Y}(x,y)\,dx\,dy
=\int_{0}^{2\pi}\int_{0}^{1}c(1-r)\,r\,dr\,d\theta
=2\pi c\int_{0}^{1}(r-r^2)\,dr
=2\pi c\left(\frac12-\frac13\right)
=\frac{\pi c}{3},
\]
getting
\[
c=\frac{3}{\pi}.
\]
Therefore
\[
f_{X,Y}(x,y)=\frac{3}{\pi}\Bigl(1-\sqrt{x^2+y^2}\Bigr)\mathbf 1_{\{x^2+y^2\le 1\}}.
\]

Computing, for $|x|\le 1$ and $a=\sqrt{1-x^2}$,
\[
f_X(x)=\int_{-\infty}^{\infty} f_{X,Y}(x,y)\,dy
=\int_{-a}^{a}\frac{3}{\pi}\Bigl(1-\sqrt{x^2+y^2}\Bigr)\,dy
=\frac{6}{\pi}\int_{0}^{a}\Bigl(1-\sqrt{x^2+y^2}\Bigr)\,dy
=\frac{6}{\pi}\left(a-\int_{0}^{a}\sqrt{x^2+y^2}\,dy\right).
\]
Using
\[
\int \sqrt{x^2+y^2}\,dy
=\frac12\Bigl(y\sqrt{x^2+y^2}+x^2\ln\bigl(y+\sqrt{x^2+y^2}\bigr)\Bigr)+C,
\]
evaluating
\[
\int_{0}^{a}\sqrt{x^2+y^2}\,dy
=\frac12\Bigl(a\sqrt{x^2+a^2}+x^2\ln(a+\sqrt{x^2+a^2})-x^2\ln|x|\Bigr)
=\frac12\Bigl(a+x^2\ln(1+a)-x^2\ln|x|\Bigr),
\]
therefore, for $|x|\le 1$,
\[
f_X(x)=\frac{6}{\pi}\left(a-\frac12\Bigl(a+x^2\ln(1+a)-x^2\ln|x|\Bigr)\right)
=\frac{3}{\pi}\left(a-x^2\ln\!\frac{1+a}{|x|}\right),
\qquad a=\sqrt{1-x^2},
\]
and
\[
f_X(x)=0,\qquad |x|>1.
\]
Taking limit at x=0
\[
f_X(0)=\lim_{x\to 0}\frac{3}{\pi}\left(\sqrt{1-x^2}-x^2\ln\!\frac{1+\sqrt{1-x^2}}{|x|}\right)=\frac{3}{\pi}.
\]
Computing analogously, for $|y|\le 1$ and $b=\sqrt{1-y^2}$,
\[
f_Y(y)=\frac{3}{\pi}\left(b-y^2\ln\!\frac{1+b}{|y|}\right),
\qquad b=\sqrt{1-y^2},
\qquad
f_Y(y)=0\ \text{for } |y|>1.
\]

Computing, for |x|<1,
\[
f_x(y)=f_{Y|X=x}(y)=\frac{f_{X,Y}(x,y)}{f_X(x)}
=\frac{\frac{3}{\pi}\left(1-\sqrt{x^2+y^2}\right)\mathbf 1_{\{x^2+y^2\le 1\}}}{\frac{3}{\pi}\left(\sqrt{1-x^2}-x^2\ln\!\frac{1+\sqrt{1-x^2}}{|x|}\right)}
=\frac{1-\sqrt{x^2+y^2}}{\sqrt{1-x^2}-x^2\ln\!\frac{1+\sqrt{1-x^2}}{|x|}}\,
\mathbf 1_{\{|y|\le \sqrt{1-x^2}\}},
\]
and $f_x(y)=0$ otherwise. Computing, for $|y|<1$,
\[
f_y(x)=f_{X|Y=y}(x)=\frac{f_{X,Y}(x,y)}{f_Y(y)}
=\frac{1-\sqrt{x^2+y^2}}{\sqrt{1-y^2}-y^2\ln\!\frac{1+\sqrt{1-y^2}}{|y|}}\,
\mathbf 1_{\{|x|\le \sqrt{1-y^2}\}},
\]
and $f_y(x)=0$ otherwise.

Checking independence
\[
f_{X,Y}(0,0)=\frac{3}{\pi},
\qquad
f_X(0)=\frac{3}{\pi},
\qquad
f_Y(0)=\frac{3}{\pi},
\qquad
f_X(0)f_Y(0)=\frac{9}{\pi^2}\ne \frac{3}{\pi},
\]
getting X,Y depending.

Computing
\[
\mathbb E[X]=\iint_{\mathbb R^2} x f_{X,Y}(x,y)\,dx\,dy
=\iint_{\mathbb R^2} (-x) f_{X,Y}(-x,y)\,dx\,dy
=-\mathbb E[X]
\Longrightarrow \mathbb E[X]=0,
\]
\[
\mathbb E[Y]=\iint_{\mathbb R^2} y f_{X,Y}(x,y)\,dx\,dy
=\iint_{\mathbb R^2} (-y) f_{X,Y}(x,-y)\,dx\,dy
=-\mathbb E[Y]
\Longrightarrow \mathbb E[Y]=0,
\]
\[
\mathbb E[XY]=\iint_{\mathbb R^2} xy f_{X,Y}(x,y)\,dx\,dy
=\iint_{\mathbb R^2} (-x)y f_{X,Y}(-x,y)\,dx\,dy
=-\mathbb E[XY]
\Longrightarrow \mathbb E[XY]=0,
\]
therefore
\[
\mathrm{Cov}(X,Y)=\mathbb E[XY]-\mathbb E[X]\mathbb E[Y]=0,
\qquad
\rho_{XY}=0.
\]

\[
\boxed{
\begin{aligned}
&f_{X,Y}(x,y)=\frac{3}{\pi}\Bigl(1-\sqrt{x^2+y^2}\Bigr)\mathbf 1_{\{x^2+y^2\le 1\}},\\[2mm]
&f_X(x)=\frac{3}{\pi}\left(\sqrt{1-x^2}-x^2\ln\!\frac{1+\sqrt{1-x^2}}{|x|}\right)\mathbf 1_{\{|x|\le 1\}},\\[2mm]
&f_Y(y)=\frac{3}{\pi}\left(\sqrt{1-y^2}-y^2\ln\!\frac{1+\sqrt{1-y^2}}{|y|}\right)\mathbf 1_{\{|y|\le 1\}},\\[2mm]
&f_x(y)=\frac{1-\sqrt{x^2+y^2}}{\sqrt{1-x^2}-x^2\ln\!\frac{1+\sqrt{1-x^2}}{|x|}}\,\mathbf 1_{\{|x|<1,\ |y|\le \sqrt{1-x^2}\}},\\[2mm]
&f_y(x)=\frac{1-\sqrt{x^2+y^2}}{\sqrt{1-y^2}-y^2\ln\!\frac{1+\sqrt{1-y^2}}{|y|}}\,\mathbf 1_{\{|y|<1,\ |x|\le \sqrt{1-y^2}\}},\\[2mm]
&X,Y\ \text{depending},\qquad \mathrm{Cov}(X,Y)=0,\qquad \rho_{XY}=0.
\end{aligned}
}
\]

\section{Problem 6}

\textbf{Problem.} Let $X$ and $Y$ be continuous random variables with a (spherically symmetric) joint PDF of the form $f(x,y)=g(x^2+y^2)$ for some function $g$. Let $(R,\theta)$ be the polar coordinates of $(X,Y)$, so that $R^2=X^2+Y^2$ is the squared distance from the origin and $\theta$ is the angle $\in[0,2\pi)$, with $X=R\cos\theta$, $Y=R\sin\theta$.

a) Prove that $R$ and $\theta$ are independent and explain intuitively why this result makes sense;

b) What is the joint PDF of $(R,\theta)$ if $(X,Y)$ is Uniform on the unit disk, i.\ e.\ $x^2+y^2\leq 1$? If $X,Y$ are i.\ i.\ d.\ $N(0,1)$?

\textbf{Solution.}

Letting $x=r\cos\theta$, $y=r\sin\theta$, $r\ge 0$, $\theta\in[0,2\pi)$.
Computing the Jacobian,
\[
\left|\frac{\partial(x,y)}{\partial(r,\theta)}\right|
=
\begin{vmatrix}
\cos\theta & -r\sin\theta\\
\sin\theta & r\cos\theta
\end{vmatrix}
=r,
\qquad dx\,dy=r\,dr\,d\theta.
\]
Substituting into the change of variables formula,
\[
f_{R,\Theta}(r,\theta)
=f_{X,Y}(r\cos\theta,r\sin\theta)\left|\frac{\partial(x,y)}{\partial(r,\theta)}\right|
=g\!\left((r\cos\theta)^2+(r\sin\theta)^2\right)\,r
=g(r^2)\,r,
\qquad r\ge 0,\ \theta\in[0,2\pi).
\]
Computing the normalization integral,
\[
1=\int_0^{2\pi}\int_0^\infty f_{R,\Theta}(r,\theta)\,dr\,d\theta
=\int_0^{2\pi}\int_0^\infty g(r^2)\,r\,dr\,d\theta
=2\pi\int_0^\infty g(r^2)\,r\,dr,
\]
\[
\int_0^\infty g(r^2)\,r\,dr=\frac{1}{2\pi}.
\]
Computing the marginal of $\Theta$,
\[
f_\Theta(\theta)=\int_0^\infty f_{R,\Theta}(r,\theta)\,dr
=\int_0^\infty g(r^2)\,r\,dr
=\frac{1}{2\pi},
\qquad \theta\in[0,2\pi).
\]
Computing the marginal of $R$,
\[
f_R(r)=\int_0^{2\pi} f_{R,\Theta}(r,\theta)\,d\theta
=\int_0^{2\pi} g(r^2)\,r\,d\theta
=2\pi r g(r^2),
\qquad r\ge 0.
\]
Factoring,
\[
f_R(r)f_\Theta(\theta)=\bigl(2\pi r g(r^2)\bigr)\left(\frac{1}{2\pi}\right)=g(r^2)\,r=f_{R,\Theta}(r,\theta),
\]
getting $R$ and $\Theta$ independent.
Interpreting $f(x,y)=g(x^2+y^2)$ as rotation-invariant, getting $\Theta$ uniform and $R$ radial.

Substituting the uniform-on-disk density,
\[
f_{X,Y}(x,y)=\frac{1}{\pi}\mathbf 1_{\{x^2+y^2\le 1\}}
=\frac{1}{\pi}\mathbf 1_{\{0\le r\le 1\}},
\]
\[
f_{R,\Theta}(r,\theta)=\frac{r}{\pi}\mathbf 1_{\{0\le r\le 1\}}\mathbf 1_{\{0\le\theta<2\pi\}}.
\]
Integrating for marginals,
\[
f_\Theta(\theta)=\int_0^1 \frac{r}{\pi}\,dr=\frac{1}{2\pi},\qquad 0\le\theta<2\pi,
\]
\[
f_R(r)=\int_0^{2\pi}\frac{r}{\pi}\,d\theta=2r,\qquad 0\le r\le 1.
\]

Substituting the i.i.d.\ $N(0,1)$ density,
\[
f_{X,Y}(x,y)=\frac{1}{2\pi}\exp\!\left(-\frac{x^2+y^2}{2}\right)
=\frac{1}{2\pi}\exp\!\left(-\frac{r^2}{2}\right),
\]
\[
f_{R,\Theta}(r,\theta)=\frac{r}{2\pi}\exp\!\left(-\frac{r^2}{2}\right)\mathbf 1_{\{r\ge 0\}}\mathbf 1_{\{0\le\theta<2\pi\}}.
\]
Integrating for marginals,
\[
f_\Theta(\theta)=\int_0^\infty \frac{r}{2\pi}e^{-r^2/2}\,dr=\frac{1}{2\pi},\qquad 0\le\theta<2\pi,
\]
\[
f_R(r)=\int_0^{2\pi}\frac{r}{2\pi}e^{-r^2/2}\,d\theta=r e^{-r^2/2},\qquad r\ge 0.
\]

\[
\boxed{
\begin{aligned}
&f_{R,\Theta}(r,\theta)=g(r^2)\,r,\qquad r\ge 0,\ 0\le\theta<2\pi,\\[2mm]
&f_{R,\Theta}(r,\theta)=\frac{r}{\pi}\mathbf 1_{\{0\le r\le 1\}}\mathbf 1_{\{0\le\theta<2\pi\}},\\[2mm]
&f_{R,\Theta}(r,\theta)=\frac{r}{2\pi}\exp\!\left(-\frac{r^2}{2}\right)\mathbf 1_{\{r\ge 0\}}\mathbf 1_{\{0\le\theta<2\pi\}}.
\end{aligned}
}
\]

\end{document}
