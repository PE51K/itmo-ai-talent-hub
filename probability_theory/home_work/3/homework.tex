\documentclass{article}

\usepackage[utf8]{inputenc}
\usepackage{amsmath}
\usepackage{amsfonts}
\usepackage{amssymb}

\title{Probability Theory Homework 3}
\author{Gregory Matsnev}
\date{\today}

\begin{document}

\maketitle

\section{Problem 1}

\textbf{Problem.} Let $X \sim \mathrm{Unif}(-1,3)$. Find the PDF of $Y = X^{2}$.

\textbf{Solution.}
The density of \(X\) is \(f_X(x)=\tfrac14\) on \([-1,3]\). 
For \(0<y<1\), we have
\[
F_Y(y)=\Pr(X^2\le y)=\Pr(-\sqrt{y}\le X\le\sqrt{y})=\int_{-\sqrt{y}}^{\sqrt{y}}\tfrac14\,dx=\tfrac{2\sqrt{y}}{4}=\tfrac{\sqrt{y}}{2}.
\]
Hence
\[
f_Y(y)=\frac{d}{dy}\left(\tfrac{\sqrt{y}}{2}\right)=\tfrac{1}{4\sqrt{y}}.
\]
For \(1<y<9\), only \(+\sqrt{y}\) lies in \([-1,3]\), so
\[
F_Y(y)=\Pr(-1\le X\le\sqrt{y})=\int_{-1}^{\sqrt{y}}\tfrac14\,dx=\tfrac{1+\sqrt{y}}{4}.
\]
Hence
\[
f_Y(y)=\frac{d}{dy}\left(\tfrac{1+\sqrt{y}}{4}\right)=\tfrac{1}{8\sqrt{y}}.
\]

\section{Problem 2}

\textbf{Problem.} Let $X \sim \mathrm{Bin}(n, p)$. Find the PMF of $Y = e^{X}$.

\textbf{Solution.}
The random variable \(Y\) takes the discrete values \(e^{k}\) for \(k=0,1,\dots,n\).
Since \(Y=e^{X}\) is a bijection on the support of \(X\),
\[
\Pr(Y=e^{k})=\Pr(X=k)=\binom{n}{k}p^{k}(1-p)^{n-k},\qquad k=0,1,\dots,n.
\]

\section{Problem 3}

\textbf{Problem.} Find the PMF of the distribution of a random variable $X$, which is equal to the number of failures in a series of Bernoulli trials with success probability $p$. Trials are carried out not a fixed number of times, but instead until there are $r$ successes. Do you remember what this distribution of a random variable is called? \textit{Note: be careful with the binomial coefficient.}

\textbf{Solution.}
To stop with exactly \(x\) failures, the sample must consist of \(x+r-1\) trials containing \(r-1\) successes and \(x\) failures, followed by a final success.
The number of ways to position the first \(r-1\) successes is \(\binom{x+r-1}{r-1}\).
Therefore
\[
\Pr(X=x)=\binom{x+r-1}{r-1}(1-p)^{x}p^{r},\qquad x=0,1,2,\dots
\]
This is the negative binomial distribution with parameters \(r\) and \(p\).

\section{Problem 4}

\textbf{Problem.} An airline overbooks a flight, selling more tickets for the flight than there are seats on the plane (figuring that it's likely that some people won't show up). The plane has 100 seats, and 110 people have booked the flight. Each person will show up for the flight with probability $0.9$, independently. Find the probability that there will be enough seats for everyone who shows up for the flight.

\textbf{Solution.}
Let \(S\) be the number of passengers who show up. Then \(S\sim\mathrm{Bin}(110,0.9)\), and seats suffice precisely when \(S\le100\).
Hence
\[
\Pr(S\le100)=\sum_{k=0}^{100}\binom{110}{k} (0.9)^{k}(0.1)^{110-k}\approx 0.6710.
\]

\section{Problem 5}

\textbf{Problem.} Prove that if $X \sim \mathrm{HGeom}(w,b,n)$ and $N=w+b \to \infty$ such that 
$p=\dfrac{w}{w+b}$ remains fixed, then the PMF of $X$ converges to the $\mathrm{Bin}(n,p)$ PMF.

\textbf{Solution.}
For \(k=0,1,\dots,n\),
\[
\Pr(X=k)=\frac{\binom{w}{k}\binom{b}{n-k}}{\binom{N}{n}}.
\]
Writing the binomial coefficients with falling factorials \((a)_m=a(a-1)\cdots(a-m+1)\) yields
\[
\Pr(X=k)=\binom{n}{k}
\left[\prod_{i=0}^{k-1}\frac{w-i}{N-i}\right]
\left[\prod_{j=0}^{n-k-1}\frac{b-j}{N-k-j}\right].
\]
As \(N\to\infty\) with \(w/N\to p\) (hence \(b/N\to1-p\)), every factor satisfies
\(\frac{w-i}{N-i}\to p\) and \(\frac{b-j}{N-k-j}\to 1-p\).
Taking the product yields
\[
\Pr(X=k)\longrightarrow \binom{n}{k} p^{k}(1-p)^{n-k},
\]
which is exactly the PMF of \(\mathrm{Bin}(n,p)\).

\section{Problem 6}

\textbf{Problem.} Consider two independent random variables $X$ with CDF $F_X$ and $Y$ with CDF $F_Y$. Find the CDFs of the random variables $Z_1=\max(X,Y)$—meaning that for every outcome $\omega$ we have $Z_1(\omega)=\max\{X(\omega),Y(\omega)\}$—and $Z_2=\min(X,Y)$.

\textbf{Solution.}
For any real \(z\),
\begin{align*}
F_{Z_1}(z)
  &= \Pr(\max(X,Y)\le z) \\
  &= \Pr(X\le z,\,Y\le z) \\
  &= F_X(z)F_Y(z),
\end{align*}
because the events are independent.

\end{document}